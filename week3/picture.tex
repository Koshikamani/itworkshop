\documentclass{article}
\usepackage{graphicx}
\usepackage{subcaption}


\begin{document}
	
	
	\begin{figure}
		\centering
		\includegraphics[width=0.5\textwidth,height=10cm,width=20cm]{form.jpg}
		\caption{forest image}
		\label{fig 1}
	\end{figure}
	
	\begin{figure}[h]
		
		\begin{subfigure}{0.5\textwidth}
			\includegraphics[width=0.5\linewidth,width=7cm]{form.jpg}
			\caption{img 1}
		\end{subfigure}
	\hfill
	\begin{subfigure}{0.5\textwidth}
		\includegraphics[width=0.5\linewidth,width=7cm]{form.jpg}
		\caption{img 2}
	\end{subfigure}
	\end{figure}

	
	
	
	
\end{document}